\section{Chapter 2}

\begin{problem}{2.2}
\textbf{Show that for any positive integers n and any base b, there must be some power of b lying in the range $[n, bn]$}.
\\
Let $x = b^{\lfloor \log_b n \rfloor}$ be the smallest power of $b$ less than or equal to $n$. $bx = b^{\lfloor \log_b n \rfloor + 1} \geq b^{\log_b n} = n$. Also, since by definition $x \leq n$, $bx \leq bn$. Thus, $bx$ is in the range $[n, bn]$.
\end{problem}

\begin{problem}{2.11}
    \textbf{Show that block matrix multiplication is valid.}
    \\
    In the multiplication $XY = Z$, consider any entry $z_{ij}$ of Z.
    \[
        X = 
        \begin{bmatrix}
        A & B \\ C & D
        \end{bmatrix}, 
        Y = 
        \begin{bmatrix}
        E & F \\ G & H
        \end{bmatrix}
    \]
    Suppose $i, j < n / 2$.
    \[
        z_{ij} = \sum_{k = 1}^{n} x_{ik}y_{kj} = \sum_{k = 1}^{n / 2} x_{ij}y_{kj} + \sum_{k = n / 2 + 1}^{n} x_{ij}y_{kj} = \sum_{k = 1}^{n / 2} a_{ij}e_{kj} + \sum_{k = n / 2 + 1}^{n} b_{ij}g_{kj}
    \]
    A similar proof can be done for the other 3 entries.
\end{problem}

\newpage
\begin{problem}{2.13}
    \textbf{Find the number of binary trees on 3, 5, and 7 vertices}. \\
    $B_3 = 1, B_5 = 2, B_7 = 5.$ \\ \\
    \textbf{Find a general recurrence for $B_n$}.
    \[
        B_n = \sum_{i = 1}^{n} B_{i - 1}B_{n - i}
    \]
    \textbf{Show by induction that $B_n \in \Omega(2^n).$} \\
    We will show more specifically that $B_n \geq c 2^{n}$. \\
    The base cases hold if we choose some $c \leq 2^{-5}$:
    \[
        B_3 = 1 \geq 2^{-2}
    \]
    \[
        B_5 = 2 \geq 2^0
    \]
    \[
        B_7 = 5 \geq 2^2
    \]
    Assuming this holds for any $7 < k < n$, consider $B_n.$
    \[
        B_n = \sum_{i = 1}^{n} B_{i - 1}B_{n - i}
    \]
    \[
        \geq \sum_{i = 1}^{n} \cdot 2^{i - 1} * 2^{n - i}
    \]
    \[
        = c \sum_{i = 1}^{n} 2^{n - 1}
    \]
    \[
        = cn \cdot 2^{n - 1}
    \]
    Since $n \geq 8$,
    \[
        cn \cdot 2^{n - 1} \geq c \cdot 2^{n + 2} \geq c2^n
    \]
\end{problem}

\begin{problem}{2.15}
\textbf{Show how to implement the \texttt{split} operation (which splits numbers according to the pivot) in place.}
    \begin{lstlisting}
        def split(arr, pivot):
            n = len(arr)
            i, j = 0, n - 1
            while i < j:
                if arr[i] >= pivot and arr[j] < pivot:
                    arr[i], arr[j] = arr[j], arr[i]
                    i, j = i + 1, j - 1
                elif arr[i] >= pivot:
                    j -= 1
                elif arr[j] < pivot:
                    i += 1
                else:
                    i, j = i + 1, j - 1
    \end{lstlisting}
\end{problem}

\begin{problem}{2.24}
    \textbf{Write down the code for quicksort.}
     \begin{lstlisting}
        def quicksort(arr):
            if not arr: return []
            pivot, l, e, r = arr[0], [], [], []
            for x in arr:
                if x < pivot: l.append(x)
                elif x = pivot: e.append(x)
                else: r.append(x)
            return quicksort(l) + e + quicksort(r)
    \end{lstlisting}
    \textbf{Show that the worst-case runtime for quicksort is $\Theta(n^2)$.}
    \\
    Suppose the pivot is always the smallest/largest element. Then the array gets reduced by 1 each time. The total runtime is $1 + 2 + 3 + ... + n = \Theta(n^2)$.
    \\ \\
    \textbf{Show that the expected runtime of quicksort is $O(n \log n)$.}
    \\
    The expected runtime satisfies the recurrence relation
    \[
        T(n) \leq Bn + \frac{1}{n}\sum_{1}^{n - 1} (T(i) - T(n - i))
    \]
    \[
        = Bn + \frac{2}{n}\sum_{1}^{n - 1} T(i)
    \]
    \[
        nT(n) \leq Bn^2 + 2\sum_{1}^{n - 1} T(i)
    \]
    \[
        (n - 1)T(n - 1) \leq B(n - 1)^2 + 2\sum_{1}^{n - 2} T(i)
    \]
    Subtract the bottom equation from the top equation.
    \[
        nT(n) - (n - 1)T(n - 1) \leq 2Bn - B + 2T(n - 1)
    \]
    \[
        nT(n) \leq (n + 1)T(n - 1) + 2Bn - B
    \]
    \[
        nT(n) \leq (n + 1)T(n - 1) + 2Bn - B
    \]
    Solving the recurrence relation gives us that $T(n) = O(n \log n)$.
\end{problem}

\begin{problem}{2.29}
    \textbf{Show that Horner's rule for polynomial evaluation gives the correct answer.}
    \\
    We use induction. Our loop invariant: at the end of the $i$th loop 
    $z = a_nx^i + a_{n - 1}x^{i - 1} + ... + a_{n - i}$.
    \\
    The base case is clear: the algorithm simply returns $a_0$ at the end of the 0th loop.
    \\
    Suppose our loop invariant is true for the $(i - 1)$st loop. Consider the $i$-th loop. By our loop invariant, we know that $z = a_nx^{i - 1}+ a_{n - 1}x^{i - 2} + ... + a_{n - i + 1}$. Inside the loop, we multiply $zx$, then add $a_{i}$, giving us $z = a_nx^i + a_{n - 1}x^{i - 1} + ... + a_{n - i + 1}x + a_{n - i}$, proving our invariant.
    \\
    Applying our invariant to when $i = n$, we get that $z = a_nx^n + a_{n - 1}x^{n - 1} + ... + a_0$ at the end of the last loop, correctly evaluating our polynomial.
    \\ \\
    \textbf{How may additions and multiplications does this routine use?}
    \\
    Horner's rule uses about O(n) additions and multiplications (one per loop). This is bad for sparse polynomials, in which most of the coefficients are 0 (for example $2x^{100} + 1$).
\end{problem}

\begin{problem}{2.20}
\textbf{Show that any array of integers$x[1...n]$ can be sorted in $O(n + M)$ time, where}
    \[
    M = \max_i x_i - \min_i x_i.
    \]
    Create an array \texttt{a} of size M. Use this array to count the number of each $x_i$ you see: \texttt{a[0]} corresponds to the smallest possible integer, \texttt{a[1]} counts the the second smallest, and so on, until \texttt{a[n - 1]} which counts the largest integer. Then, simply fill a sorted array based off the counts, starting from \texttt{a[0]}.
    \\
    Note that this sort uses no comparisons, so it is not lower-bounded by $\Omega(n \log n)$.
\end{problem}
