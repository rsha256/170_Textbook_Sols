\section{Chapter 0}
\begin{problem}{0.1}
    \textbf{In each of the following situations, indicate whether $f = O(g)$, or $f =\Omega(g)$, or both (in which case $f = \Theta(g)$).}
    
    \begin{enumerate}[(a)]
        \item $n-100=\Theta(n-200)$, drop lower terms.
        % b
        \item $n^{1/2}= O(n^{2/3})$, $0.5 < 0.\bar 6\implies\displaystyle\lim_{n\to\infty}\frac{n^{1/2}}{n^{2/3}}=0$.
        % c
        \item $100n+\log{n}=\Theta(n+(\log{n})^2)$, drop lower terms, both are $\Theta(n)$.
        % d
        \item $n\log n=\Theta(10n\log 10n)$, $g$ can be written as $10n(\log10+\log n)\in\Theta(n\log n)$.
        \item $\log 2n=\Theta(\log 3n)$, as $\log 2+\log n$ and $\log 3+\log n$ are both $\Theta(\log n)$.
        % f
        \item $10\log n=\Theta(\log(n^2))$, as $\log(n^2)=2\log n$ and as $n\to\infty$ we drop coefficient factors.
        % g
        \item $n^{1.01}=\Omega(n\log^2 n)$ as $\displaystyle\lim_{n\to\infty}\frac{n^{0.01}}{\log^2 n}\to\infty$ as polynomial growth $\gg$ logarithmic growth.
        % h
        \item $\frac{n^2}{\log n}=\Omega(n(\log n)^2)$ as $\displaystyle\lim_{n\to\infty}\frac{n}{\log^2 n}\to\infty$ as polynomial growth $\gg$ logarithmic growth.
        % i
        \item $n^{0.1}=\Omega((\log n)^{10})$, as polynomial growth $\gg$ logarithmic growth.
        % j
        \item $(\log n)^{\log n}=\Omega(n/\log n)$, 
            \begin{itemize}
                \item $\displaystyle\lim_{n\to\infty} \frac{\log^{\log(n)}(n)}{n/\log{n}}
                =\lim_{n\to\infty} \frac{\log^{\log(n)+1}(n)}{n}
                \stackrel{\text{l'hopital's}}{=}\lim_{n\to\infty}\frac{\left(\frac{\mathrm d}{\mathrm{d}n}\left[\log^{\log(n)+1}(n)\right]\right)}{1}
                \to\infty$.
            \end{itemize}
        % k
        \item $\sqrt n=\Omega((\log n)^3)$, as $\sqrt n=n^{0.5}$ and polynomial growth $\gg$ logarithmic growth.
        % l
        \item $n^{1/2}=O(5^{\log_2 n})$, 
            \begin{itemize}
                \item $5=2^{\log_2 5}\implies5^{\log_2 n}
                    =(2^{\log_2 5})^{\log_2 n}
                    =(2^{\log_2 n})^{\log_2 5}
                    =n^{\log_2 5}\approx n^{2.3}\gg n^{0.5}$.
            \end{itemize}
        % m
        \item $n2^n=O(3^n)$ as $3^n = (1.5 \cdot 2)^n = 1.5^n \cdot 2^n$ and $n\ll 1.5^n$ asymptotically.
        % n
        \item $2^n=\Theta(2^{n+1})$ as $2^{n+1}=2\cdot2^{n}$ and we drop coefficients as $n\to\infty$.
        % o
        \item $n!=\Omega(2^n)$, as factorial $\gg$ exponential (can be proved via induction).
        % p
        \item $(\log n)^{\log n}=O(2^{{(\log_2 n)}^2})$, the RHS = $n^{(\log_2 n)}$ and $\log n\ll n
        \implies
        (\log n)^{\log n}\ll n^{\log n}$.
        % q
        \item $\displaystyle\sum_{i=1}^n i^k = O(n^{k+1})$, each summation term increases $\therefore n^k$ dominates and $n^k\ll n^{k+1},$ akin to $n^2\land n^3$.
    \end{enumerate}
\end{problem}

\newpage
\begin{problem}{0.2}
    \textbf{Show that, if $c$ is a positive real number, then $g(n) = 1 + c + c^2 + \cdots + c^n$ is:}

    \[
        g(n) 
        = 1 + c + c^2 + \cdots + c^n 
        = \displaystyle\sum_{i=0}^n c^i
        = \frac{c^{n+1}-1}{c-1}, 
        (\forall c\neq 1)
    \]

    \begin{enumerate}[(a)]
        \item If $c<1,$ 
        then $\displaystyle\lim_{n\to\infty}g(n)
        =\frac{0-1}{c-1}
        =\frac{1}{1-c}
        \implies \boxed{g\in\Theta(1),} \  (\forall c < 1)$.
        % b
        \item If $c=1,$ 
        then $g(n) 
        = 1 + c + c^2 + \cdots + c^n
        = 1 + 1 + 1^2 + \cdots + 1^n
        = n + 1
        \implies \boxed{g\in\Theta(n),} \  (\forall c=1)$.
        % c
        \item If $c>1,$ 
        then $\displaystyle\lim_{n\to\infty}g(n)
        =\lim_{n\to\infty}\frac{c^{n+1}}{c}
        = c^n
        \implies \boxed{g\in\Theta(c^n),} \  (\forall c > 1)$.
    \end{enumerate}
\end{problem}

\begin{problem}{0.3}
    \textbf{The Fibonacci numbers $F_0, F_1, F_2, \ldots,$ are defined by the rule:}
    \[
        F_0=0, \ \ \ \ \ 
        F_1=1, \ \ \ \ 
        F_n=F_{n-1}+F_{n-2}.
    \]

    \begin{enumerate}[(a)]
        \item \textbf{Use induction to prove that $F_n \geq 2^{0.5n}, \ (\forall n \geq 6)$}
        \\
        Base Case(s):
        $F_6 = 8 \geq 2^{0.5 \cdot 6} = 2^{\frac{6}{2}} = 2^3 = 8. \ \ F_7=13 \geq 2^{3.5}. \ \ F_8=21 \geq 2^4.$
        \\
        Inductive Hypothesis:
        Assume $F_k\geq2^{0.5k}$ for some $n=k,$ where $k\geq6$.
        \\
        Inductive Step:
        WTS: $F_k\geq2^{0.5k}
        \implies
        F_{k+1}\geq2^{0.5(k+1)}
        =2^{0.5k+0.5}$
        \\
        $F_{k+1}=F_{k}+F_{k-1}
        > 2\cdot F_{k-1}
        \geq 2\cdot 2^{0.5(k-1)}
        = 2^{0.5k-0.5+1} 
        = \boxed{2^{0.5k+0.5}} \ .$
        \qed
        
        \item \textbf{Find a constant $c < 1$ such that $F_n \leq 2^{cn}, \  (\forall n \geq 0)$.}
        \\
        $F_n = F_{n-1} + F_{n-2} 
        \leq 2^{(c(n-1)} + 2^{(c(n-2)} 
        \leq 2^{(cn)} \cdot 2^{-c} + 2^{(cn)} \cdot 2^{(-2c)} 
        \leq 2^{(cn)} (2^{-c} + (2^{-c})^2)$
        
        $\therefore (2^{-c} + (2^{-c})^2) \leq 1$. If we look at specific case of equality with the substitution $x=2^{-c}$, we get the following: $x^2 + x - 1 = 0$
        which has roots at $x = \frac{-1 \pm \sqrt(5)}{2}$ per the quadratic formula.
        \\
        Note that we take the positive root as the logarithm (used below) is only defined for inputs $>0$.
        \[
            x=2^{-c}
            \implies
            c = -\log_2 x
        \]
        \[
            \boxed{\text{This holds true }\forall c \ \text{ s.t. } 0.694242 \leq c < 1}.
        \]
        
        \item $F_n=\Omega(2^{cn})\implies\exists k\in\mathbb Z^+: F_n \geq k\cdot 2^{cn}$.
        WLOG equality holds when $c=-\log_2(\frac12 (-1 + \sqrt{5}))$
        \[
            \implies
            \boxed{c \approx 0.694242}
        \]
    \end{enumerate}
\end{problem}

\newpage
\begin{problem}{0.4}
    In general, to compute $F_n$, it suffices to raise the following $2\times 2$ matrix, called $X$, to the $n$th power:
    \[
        \begin{bmatrix}
                F_n \\
                F_{n+1}
        \end{bmatrix}
        =
        \begin{bmatrix}
                0 & 1 \\
                1 & 1
        \end{bmatrix}^n
        \cdot
        \begin{bmatrix}
                F_0 \\
                F_1
        \end{bmatrix}.
    \]
    \begin{enumerate}[(a)]
        \item \textbf{Show that two 2 × 2 matrices can be multiplied via 4 additions \& 8 multiplications:}
        \\
        For $2\times2$ matrices:
        \[
            \begin{bmatrix}
                a & b \\
                c & d
            \end{bmatrix}
            \begin{bmatrix}
                e & f \\
                g & h
            \end{bmatrix}
            =
            \begin{bmatrix}
                a\cdot e + b\cdot g & a\cdot f + b\cdot h \\
                c\cdot e + d\cdot g & c\cdot f + d\cdot h
            \end{bmatrix}
        \]
    which clearly has four additions (denoted by $+$) and eight multiplications (denoted by $\cdot$).
    
    \vspace{5 mm}
    But how many matrix multiplications does it take to compute $X^n$?

    \item \textbf{Show that $O(\log n)$ matrix multiplications suffice for computing $X^n$.}
    \\
    Repeated Squaring Algorithm:
    \[
        X^n = 
        \begin{cases}
            (X^{\frac n2})^2 &\text{if } n \text{ is even} \\
            X\cdot X^{n-1} &\text{if } n \text{ is odd}
        \end{cases}
    \]
    Note that this is applied recursively with $n$ guaranteed to be reduced by half within every $2$ iterations $\implies\text{ Repeated Squaring Algorithm }\in O(\log n)$.

    \item \textbf{Show that all intermediate results of \mintinline{python}{fib3} are $O(n)$ bits long.}
    \\
    Adding two numbers that have $n$ bits will result in at most an $n+1$ bit number.

    \item \textbf{Prove that the running time of \mintinline{python}{fib3} is $O(M(n)\log n)$.}
    \\
    There are $\log n$ multiplications, each of which is $M(n)\implies$ \mintinline{python}{fib3} $\in O(M(n)\log n)$.

    \item \textbf{Can you prove that the running time of \mintinline{python}{fib3} is $O(M(n))$?}
    \\
    Yes!
    \end{enumerate}
\end{problem}
