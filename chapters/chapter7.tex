\section{Chapter 7}

\begin{problem}{7.25}
    \textbf{Write the max flow problem as a linear program.} \\
    We represent the flow along a s-t path as a variable $x_p$.
    \[
        \max \sum_p x_p
    \]
    \[
        x_p \geq 0; \forall e
    \]
    \[
        \sum_{p: e \in p} x_p \leq c_e; \forall e
    \]
    
    \textbf{Find the dual of the LP max flow problem.}
    \[
        \min \sum_e c_ey_e
    \]
    \[
        \sum_{e \in p} y_e \geq 1
    \]
    \[
        y_e \geq 0
    \]
    
    \textbf{Interpret the dual LP.} \\
    The dual LP can be seen as finding a min cut on the graph. An edge is included in the cut iff $y_e = 1$. We minimize over the sum of included edges, and ensure that across any s-t path, at least one edge on this path is chosen.
\end{problem}

\begin{problem} {7.28}
    \textbf{Show that finding a shortest s-t path is equivalent to finding a flow of size 1 minimizing $\sum_e l_ef_e$ with no capacity constraints.} \\
    We can interpret the flow found as follows: if an edge has 1 unit of flow, include it in the shortest path; otherwise do not. Clearly, this minimizes the sum of edges along any s-t path, and finds the shortest path. \\ \\
    \textbf{Write the shortest path problem as an LP}. \\
    \[
        \min \sum_e l_ef_e
    \]
    \[
        \sum_{e=(u, v)} f_e - \sum_{e=(v, w)} f_e \geq 0, \forall v \neq s, t
    \]
    \[
        \sum_{e = (v, t)} f_e \geq 1
    \]
    
    \textbf{Find the dual of this LP.} \\
    \[
        \max x_t
    \]
    \[
        x_v - x_u \leq f_{e = (u, v)}
    \]
    \[
        x_v \geq 0
    \]
\end{problem}

\newpage
\begin{problem}{7.31}
\textbf{Show how to find the fattest s-t path using a variant of Dijkstra's algorithm.}
\begin{lstlisting}
    widest = int[v]
    queue = []
    while queue:
        v = queue.dequeue()
        for w in G[v]:
            if widest[w] < min(cap(v, w), widest[v]):
                widest[w] = min(cap(v, w), widest[v])
\end{lstlisting}

\textbf{Show that the maximum flow in $G$ is a sum of individual flows along at most $|E|$ paths from s to t.} \\
Take the edge $e = (u, v)$ with minimum flow $f_e$. If it is part of a cycle, remove $f_e$ units of flow from each edge in the cycle. If it is part of a path, remove $f_e$ units of flow from each edge in the path. This maintains conservation of flow: the net flow into $u$ is decreased by $f_e$ and so it the net outflow; similarly for $v$. Remove this edge from the graph and repeat--after at most $|E|$ iterations, all edges will be gone and the graph will be decomposed into the paths/cycles found during this process. \\ \\

\textbf{Show that increasing the flow along the fattest path allows Ford-Fulkerson to terminate in at most O($|E| \log F$) iterations}. \\
Suppose the max flow has size $|F|$. Let $F_k$ be the amount of flow left to reach the max flow, in the $k$th iteration. Since the max flow can be decomposed into at most $|E|$ paths, the fattest path must have flow greater than or equal to $|F| / |E|$. Pushing flow along this path results in $F_{k + 1} = (1 - 1 / |E|)F_k$. Using induction, we can see that $F_{k}$ reaches 0 when $k = E \log F$, meaning we terminate after that many iterations.
\end{problem}
